\documentclass{beamer}

\usepackage{default}
\usepackage{listings}

\usetheme{Boadilla}
%\usecolortheme{spruce}

\title{Predicting recipe sentiment}
\subtitle{from nutrient composition}
\author{Mike Touse}
\date{4 March 2017}

%\setbeamercolor{banner}{fg=green}

\begin{document}

\begin{frame}
	\titlepage
\end{frame}

\begin{frame}
	\frametitle{Outline}
	\tableofcontents
\end{frame}

\section{Introduction}

\begin{frame}
	\frametitle{Introduction}
	\begin{block}{Recipe success is a high-stakes game}
		\begin{itemize}
			\item 10\% of the American labor force is employed by restaurants operating on razor-thin margins.  Offering a recipe that customers will not like wastes precious resources (time and logistics).
			\item A top recipe website sees 100 million visits every month.  Predicting the success of recipes before receiving feedback allows significant advantage in marketing revenue-generating advertisement space.
		\end{itemize}  	
	\end{block}

	\begin{alertblock}{Goal:}
		Develop a model to predict recipe success, based only on nutrient quantities, to maximize extensibility across features such as ethnicity and audience.
	\end{alertblock}

\end{frame}



\section{Data Collection}

\begin{frame}
	\frametitle{Data collection and distribution}
	\begin{columns}[t]
		\column[T]{5cm}
		\begin{block}{Process}
			\begin{itemize}
				\item Query subsets of recipes using individual ingredients to get ratings and recipe id numbers.
				\item Concatenate subsets into single list of recipes.
				\item Query individual recipes from compiled list to capture complete recipe information.
				\item Parse necessary fields into individual components (nutrients).
			\end{itemize}
		\end{block}
		
		\column[T]{5cm}

		\only<1>{Original data distribution:}
		\includegraphics<1>[height=3cm]{InitialDistro.png}

		\only<2>{\alert{Final} dataset distribution:}
		\includegraphics<2->[height=3cm]{FinalDistro.png}
		
		\begin{block}{Unbalanced ratings}
			\begin{itemize}
				\item <1-> Initial API responses mostly rated 4
				\item <2-> Adjusted API script to balance dataset
				\item <2-> Grouped ratings in Favorable (4-5) and Unfavorable (0-3)
			\end{itemize}
			
		\end{block}
				
	\end{columns}
	
	
\end{frame}

\begin{frame}{Recipe response to JSON query}
		
	\begin{block}{Each includes only nutrients present in the recipe.  Example:}
		\centering
		\includegraphics<1>[height=6cm]{APIresponse.png}
		
	\end{block}

\end{frame}

\begin{frame}{Data overview}
	\begin{block}{}
		\begin{itemize}
			\item 21,140 recipes
			\item 113 different nutrients
			\item Many sparse features
			\item Scaled to single-serving
		\end{itemize}
	\end{block}			
	\begin{block}{Sample descriptive statistics}
		\centering
		\small
		\begin{tabular}{lccccc}
			{} &  FAT\_KCAL &         K &         FLD &     PHYSTR &     VAL\_G \\
			\hline
			count &   21140 &  21140 &  21140 &  21140 &  21140 \\
			mean  &      51.6 &      0.16 &      0.0000 &      0.005 &      0.19 \\
			std   &      94.8 &      0.23 &      0.0000 &      0.015 &      0.34 \\
			min   &       0.0 &      0.00 &      0.0000 &      0.000 &      0.00 \\
			25\%   &       4.2 &      0.01 &      0.0000 &      0.000 &      0.00 \\
			50\%   &      23.3 &      0.08 &      0.0000 &      0.001 &      0.04 \\
			75\%   &      62.5 &      0.21 &      0.0000 &      0.005 &      0.28 \\
			max   &    1810.0 &      2.95 &      0.0025 &      0.630 &      5.54 \\
		\end{tabular}
		
	\end{block}
	
%	\begin{columns}
%		\column[T]{5cm}
%		\column[T]{5cm}
		 
%	\end{columns}
	
\end{frame}



\section{Data Exploration}


\begin{frame}
	\frametitle{Data Exploration}
	\begin{columns}[t]
		\column[T]{5cm}
		\begin{block}{Observations}
			\begin{itemize}
				\item Positive and negative ratings are largely overlapped
				\item No linear discriminants
				\item Correlations are small and mostly negative
			\end{itemize}
		\end{block}
		\begin{block}{Correlation}
			\includegraphics[height=2.9cm]{corr.png}
		\end{block}
		
		\column[T]{5.5cm}
		\begin{block}{Illustrative example of two most-important nutrients:}
		\includegraphics[height=5cm]{smallPair.png}
		\end{block}
		
	\end{columns}
	

\end{frame}


\begin{frame}{Feature positivity}
	Evaluated portion of recipes rated positively at each decile of each nutrient
	\begin{columns}[T]
		\column[T]{7.5cm}
		\begin{block}{Fraction of positive ratings vs. Decile}
		\includegraphics[height=5cm]{Decile.png}
		\end{block}
	
		\column[T]{4cm}
		\begin{itemize}
			\item Most nutrients exhibit single decile with sharp minimum
			\item Grouped by min decile (line width is number of nutrients)
			\item Confirms negative correlation
		\end{itemize}
	\end{columns}
\end{frame}	

\section{Classifiers}

\begin{frame}{Logistic Regression (LR)}
	
	\begin{columns}
		\column[T]{4.5cm}
		\begin{block}{LR Setup}
			\begin{itemize}
				\item Scikit-learn StandardScaler
				\item L1 penalty for sparsity:  Zero-weights - little impact
				\item Minimal sensitivity to regularization parameter
				\item 80/20 train/test split
			\end{itemize}
		\end{block}
		
		\column[T]{6.5cm}
		\begin{block}{Feature Weights}
			\includegraphics[height=4.25cm]{LRweights.png}
		\end{block}	
	\end{columns}
	
\end{frame}

\begin{frame}{Logistic Regression Results}
	\begin{columns}
		\column[T]{6cm}
		\begin{block}{Performance Metrics}
			\begin{table}
				\begin{tabular}{lr}
					\textbf{Metric} & \textbf{Value} \\
					\hline
					Accuracy &  0.66 \\
					Precision & 0.64 \\
					Recall & 0.87 \\
					F1 & 0.73 \\
					Area under ROC curve & 0.70 \\	
				\end{tabular}
			\end{table}
			
		\end{block}
		
		\column[T]{4cm}
		\begin{block}{Confusion Matrix}
			\centering
			\includegraphics[height=3cm]{lr_cm.png}
		\end{block}	
		
	\end{columns}
	
		\begin{block}{Performance observations}
			\begin{itemize}
				\item High False Positive Rate (FPR) causes misleading Recall score (artificially high)
			\end{itemize}
			
		\end{block}	

	
\end{frame}



\begin{frame}{Random Forest}
	\begin{block}{Parameters}
		\begin{itemize}
			\item 10,000 estimators
			\item Gini criterion
			\item Max features: sqrt
			\item Unscaled (original per-serving data)
		\end{itemize}
	\end{block}
	
	\begin{block}{Relative feature importance (top 50)}
		\begin{columns}
			\column[T]{5cm}
			\centering
			\includegraphics[height=4cm]{50importances.png}
			
			\column[T]{3cm}
			\begin{enumerate}
				\item Water
				\item Vitamin A
				\item Starch
				\item Potassium
				\item Fiber
				\item . . .
			\end{enumerate}
		\end{columns}
		
	\end{block}	
	
\end{frame}



\begin{frame}{Random Forest Results}
	\begin{columns}
		\column[T]{6cm}
		\begin{block}{Performance Metrics}
			\begin{tabular}{lr}
				\textbf{Metric} & \textbf{Value} \\
				\hline
				Accuracy &  0.74 \\
				Precision & 0.75 \\
				Recall & 0.78 \\
				F1 & 0.76 \\
				Area under ROC curve & 0.80 \\	
			\end{tabular}
		\end{block}
		
		\column[T]{4cm}
		\begin{block}{Confusion Matrix}
			\centering
			\includegraphics[height=3cm]{confusionMatrix.png}
		\end{block}	
		
	\end{columns}
	
	\begin{block}{Performance observations}
		\begin{itemize}
			\item Balanced True Positive Rate and True Negative Rate
			\item 10\% increase across all metrics (except Recall) relative to LR classifier
		\end{itemize}
		
	\end{block}	
	

	
\end{frame}




\section{Conclusion}

\begin{frame}{Conclusion}
	
	\begin{block}{Results}
		\begin{itemize}
			\item Correctly predicted positive recipe ratings approximately \alert{75\%} of the time (depending on metric used) using \alert{only} nutrients.
			
		\end{itemize}		
	\end{block}	

	\begin{block}{Recommendations}
		\begin{itemize}
			\item Restaurants should apply predictive model to recipes prior to preparing them for customers and avoid costly logistics (and potential customer turn-off) while collecting feedback.
			\item Recipe websites should use prior probabilities from model to tune advertisement sales algorithms.
			\item Recipe developers should use model discoveries to avoid overuse of specific nutrients.
			\item Despite mediocre predictive performance, favor inclusion of theobromine and caffeine as the nutrients found most positively correlated with favorability (and heavily concentrated in chocolate).
		\end{itemize}		
	\end{block}	
			
\end{frame}

%\begin{frame}
%	\frametitle{What’s Still To Do?}
%	\begin{block}{Answered Questions}
%		How many primes are there?
%	\end{block}
%	\begin{block}{Open Questions}
%		Is every even number the sum of two primes?
%	\end{block}
%
%	\begin{definition}
%		A \alert{prime number} is a number that has exactly two divisors.
%	\end{definition}
%	
%	\begin{example}
%		\begin{itemize}
%			\item 2 is prime (two divisors: 1 and 2).
%			\item 3 is prime (two divisors: 1 and 3).
%			\item 4 is not prime (\alert{three} divisors: 1, 2, and 4).
%		\end{itemize}
%	\end{example}
%
%\end{frame}

\end{document}
