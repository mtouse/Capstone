\documentclass[]{scrartcl}
\usepackage{graphicx}
\usepackage{longtable}

%opening
\title{Predicting recipe sentiment from nutrient composition}

\author{Mike Touse}

\begin{document}

\maketitle

\begin{abstract}
The goal of this project is to develop a model to predict the popularity of a given recipe to allow both restaurants and recipe publishers to better select content for their recipe collections before launching into costly customer trials and complicating their menu and associated supply chain requirements.  In order to maximize extensibility into various markets, the model is developed using nutrient content as a basis (as opposed to ingredients, ethnicity, or other high-level features).  After supervised training of the model with only nutrient content and customer ratings, new recipes were classified as either favorable or unfavorable with 75\% precision and an F1 score of 76\%.  
\end{abstract}

\section*{Introduction}
Upserve\footnote{https://upserve.com/} and Avero\footnote{http://www.averoinc.com/} are among a number of restaurant management resources that help restaurants manage everything from employee schedules, customer history, and sales analytics.  According to Upserve's website
\begin{quote}
	Every day, 14 million Americans show up to work at a restaurant. That’s 10\% of the U.S. workforce. Almost all restaurants have fewer than 50 employees, and most are independent, not part of a giant chain. The restaurateurs leading these businesses run on tight margins. They don’t have a big capital budget. They aren’t software gurus. And most of all, they don’t have much time to get a handle on everything – guests, staff, menu, marketing and finance – making it harder than it has to be to take their restaurant to the next level.\footnote{https://upserve.com/company/}
	\end{quote}

One particular feature of Upserve's platform, which serves more than 7,000 restaurants nationwide, is `Menu Intelligence' which couples sales data with menu items at the individual-customer level.  Because sales trends are a lagging indicator, such management platforms (and their customers) could benefit tremendously from increased pre-launch confidence in recipe success.  

Similarly, recipe websites generate revenue through either subscription fees or advertisement sales and rely heavily on customer satisfaction with the recipes presented.  With nearly 100 million monthly site visits to a top recipe website\footnote{https://www.similarweb.com/website/cookpad.com}, improving site membership and click-through-rate (CTR) can generate significant revenue for minimal investment.  Predicting recipe performance even before receiving user ratings would allow the company to highlight new recipes with reasonable confidence that the recipe would improve overall site utilization and yield higher revenue-generating CTRs.

\section*{Data Collection}
In order to develop a precise and accurate model of recipe success, a collection of recipe nutrients and a customer rating of those recipes was required.  Yummly.com uses two different Application Programming Interfaces (APIs) to provide responses in JSON format:
\begin{itemize}
	\item Search Recipe - returns a list of recipes that meet the search criteria and provide some attributes of the recipe  
	\item Get Recipe - uses a single ID provided from the Search Recipe response to return the detailed nutritional data and recipe text details
\end{itemize}   

Because each recipe must be downloaded individually once the list of recipes to download was selected, scripts were developed to handle the API requests within the education license term limits through the following process:
\begin{itemize}
	\item Query subsets of recipes using individual ingredients (such as beef, chicken, tomatoes, etc).  Responses included ratings and recipe id's.
	\item Concatenate subsets into single list of recipes.
	\item Query individual recipe id's from concatenated list to capture complete recipe information.
	\item Parse necessary fields into individual components (namely, the quantity of individual nutrients).
\end{itemize}

Following an initial collection of the recipe list, it became apparent that the recipe ratings were highly unbalanced (with nearly all recipes rated 4/5 and a smaller percentage rated 3/5 or 5/5 as seen in Figure \ref{fig:InitDistro}).  Further investigation suggested that selecting recipes towards the end of the API response tended towards lower ratings and the bulk collection scripts were adapted to balance the rating distribution of the dataset to allow improved model training.  

\begin{figure}
	\includegraphics{InitialDistro.png}
	\caption{Distribution of recipe ratings following initial download.\label{fig:InitDistro}}
\end{figure}


Even after adding lower-ranked data to the set, most of the recipes were rated either 3/5 or 4/5.  This would likely give poor results if attempting to predict actual recipe ratings, so the goal was shifted to a binary classification of either 'Favorable' (4/5 or 5/5) or 'Unfavorable' (0/5 - 3/5) whose final distribution is shown in Figure \ref{fig:FinalDistro}.  

\begin{figure}
	\includegraphics{FinalDistro.png}
	\caption{Distribution of recipe ratings used for model development.  Note roughly equal shares of Favorable vs Unfavorable ratings.\label{fig:FinalDistro}}
\end{figure}


\section*{Data Structure}

Once the recipe list was compiled using the Search Recipe API, the individual recipes were downloaded and the JSON responses were ingested into a Pandas DataFrame.  The nutrient information was contained in a list within a single column and required further parsing.  Ultimately, a DataFrame was compiled that contained the recipe ID, rating, number of servings, preparation time, and individual nutrients, each as a separate feature (column).  The recipe response only populates fields for the nutrients listed for the individual recipe, so the DataFrame was constructed to include every nutrient listed, with many of the fields populated with null ('NaN') values.  Nutrient values were normalized to a 'per serving' value using the 'number of servings' field, and null elements were filled with a value of '0'. 

\section*{Data Exploration}

The final dataset after cleaning included 21,140 samples (recipes) comprised of 113 nutrients (with many nutrients not present in all recipes).  A sampling of five nutrients are presented in Table \ref{tab:Describe} along with some descriptive statistics such as mean, standard deviation, and quartile values.  The list of all possible nutrients and their abbreviations are presented in the Appendix (Table \ref{App:NutTable}).

\begin{table}
	\centering
	\caption{Descriptive statistics of a sampling of features from final dataset.  It is important to note that many features remain 0.0 through multiple quartiles, indicating sparsity of many nutrients.\label{tab:Describe}}
	\begin{tabular}{lrrrrr}
		{} &      FAT\_KCAL &             K &           FLD &        PHYSTR &        VAL\_G \\
		\hline
		mean  &      51.6 &      0.16 &      0.0000 &      0.005 &      0.19 \\
		std   &      94.8 &      0.23 &      0.0000 &      0.015 &      0.34 \\
		min   &       0.0 &      0.00 &      0.0000 &      0.000 &      0.00 \\
		25\%   &       4.2 &      0.01 &      0.0000 &      0.000 &      0.00 \\
		50\%   &      23.3 &      0.08 &      0.0000 &      0.001 &      0.04 \\
		75\%   &      62.5 &      0.21 &      0.0000 &      0.005 &      0.28 \\
		max   &    1810.0 &      2.95 &      0.0025 &      0.630 &      5.54 \\
	\end{tabular}
		
\end{table}

Initial exploratory data analysis focused on examining which of the nutrients were reported in the largest percentage of recipes and the distribution of ratings across each feature in order to identify potential predictive features.  Results from this analysis revealed very little correlation between any particular nutrient and the rating/class label, as illustrated in Figure \ref{fig:corr} which shows the correlation coefficient across all features, with the vast majority of nutrients negatively correlated with the recipe classification.  The maximum (absolute) correlation coefficient was found to be 0.18 for K (Potassium, which is negatively correlated).

\begin{figure}
	\begin{center}
	\includegraphics[scale=.8]{corr.png}
	\caption{Correlation coefficient of all nutrients with positive recipe classification.  Note low levels of correlation and negative skew.} \label{fig:corr}
	\end{center}
\end{figure}

%\begin{table}
%	\centering
%	\caption{Correlation of most significant nutrients (5 positive and 5 negative) with the recipe classification.\label{tab:Correlation}}
%	\begin{tabular}{lr}
%	\textbf{Nutrient} & \textbf{Correlation} \\
%	\hline
%	THEBRN           &  0.040 \\
%	CAFFN            &  0.019 \\
%	Vitamin E, added &  0.016 \\
%	HYP              &  0.013 \\
%	GALS             &  0.013 \\
%	FE      & -0.155 \\
%	NIA     & -0.168 \\
%	WATER   & -0.169 \\
%	VITC    & -0.169 \\
%	K       & -0.179 \\
%	\end{tabular}
%\end{table}

The lack of correlation between nutrients can be seen even more clearly in Figure \ref{fig:pairplot}.  The nutrients shown were determined to be some of the most important features in the classification methods used in following sections and demonstrate the lack of any separating plane between the positively rated (blue) and negatively rated (orange) recipes, likely preventing the use of linear discriminants or support vectors for classification.  The visual presentation in Figure \ref{fig:pairplot} does, however, support the observation in Figure \ref{fig:corr}  that most nutrient quantities are negatively correlated with recipe classification.

\begin{figure}
	\includegraphics[scale=.45]{pairplot.png}
	\caption{Scatter plots of several nutrient pairs (blue points are highly rated recipes, orange are poorly rated).  Plots shown are illustrative of the data distribution across the entire dataset.  Diagonal plots show histogram of the given feature.} \label{fig:pairplot}
\end{figure}


Each nutrient was then examined to determine whether a particular value  of that nutrient (per serving) would yield a significant favorability descriminant.  By viewing the fraction of positive ratings in each decile, it was noted that most nutrients followed a downward trend with increasing nutrient content but with a significant minimum within a particular decile.  Nutrients were then grouped by the location of their minimum favorability decile and plotted as the portion of positive ratings in each decile versus the actual decile.  Line widths demonstrate the number of nutrients in each grouping which indicates that many recipes become unfavorable as the nutrient quantities increase above roughly their fifth or sixth decile, though many others demonstrate minima at lower deciles.

\begin{figure}
	\centering
	\includegraphics[scale=.8]{Decile.png}
	\caption{Fraction of recipes that are highly rated at each decile, by nutrient, grouped according to decile that contains the minimum favorability.  Line widths indicate number of nutrients in each group. \label{fig:Deciles}}
\end{figure}

\section*{Classification}
\subsection*{Logistic Regression}
In order to identify weights of particular features that can be used to classify individual recipes, the first step was to attempt to fit a logistic regression to the nutrient data.  Because of the large number of features and relative sparsity of some, L1 regularization was expected to yield more useful results by driving a portion of individual feature weights to zero.  

\begin{table}
	\centering
	\caption{Performance metrics of logistic regression classification.\label{tab:lr_metrics}}
	\begin{tabular}{lr}
		\textbf{Metric} & \textbf{Value} \\
		\hline
		Accuracy &  0.66 \\
		Precision & 0.64 \\
		Recall & 0.87 \\
		F1 & 0.73 \\
		Area under ROC curve & 0.70 \\	
	\end{tabular}
\end{table}
\begin{figure}
	\centering
	\includegraphics[scale=.70]{lr_cm.png}
	\caption{Confusion matrix of predicted classification of test samples following logistic regression classifier supervised training. Labels of '0' indicate low ratings and labels of '1' indicate high ratings.\label{fig:lr_cm}}
\end{figure}

\begin{figure}
	\centering
	\includegraphics[scale=.75]{lr_roc.png}
	\caption{Receiver Operating Characteristic curve showing True Positve Rate versus False Positive Rate following logistic regression classification.  \label{fig:lr_roc}}
\end{figure}

To prepare the data to develop a logistic regression classifier, the data was scaled by removing the mean and scaling to unit variance, and then split into randomly sampled training (80\%) and test (20\%) recipes.  Initial classification with Logistic Regression was attempted with C=1 and L1 regularization with successful convergence and acceptable results. Several other values of C were tested with little impact to accuracy and other metrics. L2 regularization was also tested with little impact. 



The logistic regression classifier was able to predict recipe classification with an F1 score of 0.73.  This and other metrics are shown in Table \ref{tab:lr_metrics}.  As can be seen in the confusion matrix (Figure \ref{fig:lr_cm}), the logistic regression classifier suffered from a significant number of false positive classifications.  While the developed model would offer slight benefit to the example restaurateur or recipe author, better performance was expected through ensemble methods such as a Random Forest classifier which is reported in the next section.  

\subsection*{Random Forest Classifier}
\begin{figure}
	\centering
	\includegraphics[scale=.8]{50importances.png}
	\caption{Fifty most important features identified in Random Forest classifier.\label{fig:50imp}}
\end{figure}

Because the Random Forest classifier samples across all features and recipes to build each individual tree, the original training dataset (scaled only to single-recipe values) was used instead of the set that was demeaned and scaled for logistic regression.  Using a Random Forest Classifier with 10,000 estimators and a maximum of 11 features per tree, the classifier was trained and feature importance was determined, the top 50 of which are shown in Figure \ref{fig:50imp}. 

\begin{table}
	\centering
	\caption{Performance metrics of Random Forest classification. \label{tab:rf_metrics}}
	\begin{tabular}{lr}
		\textbf{Metric} & \textbf{Value} \\
		\hline
		Accuracy &  0.74 \\
		Precision & 0.75 \\
		Recall & 0.78 \\
		F1 & 0.76 \\
		Area under ROC curve & 0.80 \\	
	\end{tabular}
\end{table}

After fitting the training data, the test data was again used to predict the ratings and compared to the true ratings from the original dataset. Similar to the logistic regression classifier reported above, the performance of the Random Forest classifier was evaulated using the metrics reported in Table \ref{tab:rf_metrics}.  Significant improvement was seen as compared to the performance of the logistic regression, with accuracy of 74\%.  All metrics improved with the exception of recall, which benefitted from the high False Positive Rate in the logistic regression classifier by capturing a large percentage (87\%) of the positives with its high positive bias.

\begin{figure}
	\centering
	\includegraphics[scale=.70]{confusionMatrix.png}
	\caption{Confusion matrix of predicted classification of test samples following Random Forest classifier supervised training. Labels of '0' indicate low ratings and labels of '1' indicate high ratings.\label{fig:CM}}
\end{figure}
Examination of the confusion matrix (Figure \ref{fig:CM}) shows that the correctly predicted recipes are much more evenly distributed between true positive (1, or high ratings) and true negative (0, or low ratings), and correct predictions far outnumber incorrect as seen in the metrics above.  Additionally, the ROC curve in Figure \ref{fig:ROC} shows a better balance of True and False Positive Rates, yielding an area of 0.80 under the ROC curve.
\begin{figure}
	\centering
	\includegraphics[scale=.75]{ROC.png}
	\caption{Receiver Operating Characteristic (ROC) curve of Random Forest classifier output.  Area Under ROC Curve (AUC) = 0.80.\label{fig:ROC}}
\end{figure}

\begin{figure}
	\centering
	\includegraphics[scale=.8]{Trees.png}
	\caption{Feature importances for top 50 features from Random Forest classifier (blue), with importances from a single test tree with high probability of positive rating (orange).  Actual importance values are relative and, therefore, not shown.\label{fig:Trees}}
\end{figure}

Because the random forest classifier aggregates a large set of decision trees, it is important to illustrate that the relative importance values do not equate to relative concentrations, nor do they identify the relative importances of any particular decision tree.  In Figure \ref{fig:Trees}, the importance values of the individual nutrients in a single recipe are presented with the top 50 aggregate importances shown previously in Figure \ref{fig:50imp}.  Note that the (orange) individual tree values show a similar trend across the aggregate values, but individual nutrient importances differ greatly when considered in a single recipe.

Finally, it is worth discussing overall observations of the nutrients identified as important for classifying recipe likability.  As discussed earlier, the most strongly identified nutrients have negative correlations, meaning that excess amounts of those nutrients are most likely to cause a recipe to be rated poorly.  


\begin{table}
	\centering
	\caption{Example list of recipes titles containing levels of Vitamin A corresponding to the most Favorable and most Unfavorable deciles.  Recipes in each list also correspond to the ultimate rating of the recipe and were selected by proximity to the mean value within the decile.\label{tab:VITA}}
	\begin{tabular}{ll}
	   \textbf{Favorable} &  \textbf{Unfavorable}\\
	   \hline
	   Butter-Bombs & Crockpot-Rice-Pilaf\\
	   Buttered-skillet-corn & Five-Spice-Beef\\
	   California-style-deviled-eggs & Flour-Focaccia-from-Flour-Too\\
	   Chipotle\_s-Copycat-Cilantro-Lime-Rice & Golden-Milk\\
	   Crunchy-Toasted-Pasta-Leftovers & Homemade-Buttercream-Mints\\
	   Delicious-Fresh-Cranberry-Orange-Curd & Almond-Milk-AND-Almond-Flour\\
	   Duchess-Potatoes & Knuspermusli-Wurfel\\
       Easy-Curry-Rice-Allrecipes & Oven-Baked-Healthier-Fish-Cakes\\
       Fruit-and-Nut-Granola-Cookies & Roast-Pork\\
       Garlicky-Steak-and-Grilled-Corn &  Rum-Balls\\
	\end{tabular}
\end{table}


Some very familiar nutrients stand out as being significant and negatively correlated, namely water, potassium, vitamin C, and vitamin A.  Another observation is a rough grouping of the amino acids such as alanine, serine, histidine, and glutamic acid near the top third of the identified nutrients (again, negatively correlated).  While no causation stands out that explains the observations, one hypothesis is that some of these nutrients might cause a recipe to be assessed as too strong (vitamins A and C) or too diluted (water) causing harsher critique by the rater.  

Recipes that contained the most favorable and least favorable quantities according to the previous decile analysis were examined more closely to identify recipe characteristics that might explain their favorability.  As seen in Table \ref{tab:VITA}, favorable recipes in the most favorable decile of Vitamin A quantities span all categories of recipes and courses including deserts, side dishes, main courses, and even condiments.  A similar broad representation is seen in the unfavorable recipes in the most unfavorable decile.  This analysis emphasizes the broad applicability of the model, but also highlights an area of future work to evaluate the appropriate ratios of nutrients for recipe favorability as well as a mapping of ingredients to nutrients that would help recipe developers intelligently craft their creations.

\section*{Conclusion}

In the analysis presented, a set of rated recipes and their nutrients was collected, cleaned, and evaluated for potential discriminating features to allow prediction of user rating.  The data was then used to ultimately train a Random Forest classifier which correctly predicted user rating with 74\% accuracy.  

Preselecting recipes, either as a restaurateur or recipe author, based on a reasonable certainty of their success is obviously advantageous in terms of time and resource efficiency.  An additional advantage of the analysis and supporting data is that individual nutrients were identified that can provide the user guidelines within which they can develop recipes that have, at a minimum, 74\% chance of being rated favorably by users or customers.  

The closing advice to a recipe developer would be to strongly consider the most positively correlated nutrients: caffeine and theobromine which can both be found in high concentrations in chocolate.

\pagebreak

\part*{Appendix:  Nutrient List}
\begin{longtable}{lll}
	\textbf{Nutrient} & \textbf{Units} &     \textbf{Tagname} \\
	\hline
	Protein                            &     g &      PROCNT \\
	Total lipid (fat)                  &     g &         FAT \\
	Carbohydrate, by difference        &     g &      CHOCDF \\
	Ash                                &     g &         ASH \\
	Energy                             &  kcal &  ENERC\_KCAL \\
	Starch                             &     g &      STARCH \\
	Sucrose                            &     g &        SUCS \\
	Glucose (dextrose)                 &     g &        GLUS \\
	Fructose                           &     g &        FRUS \\
	Lactose                            &     g &        LACS \\
	Maltose                            &     g &        MALS \\
	Alcohol, ethyl                     &     g &         ALC \\
	Water                              &     g &       WATER \\
	Adjusted Protein                   &     g &         NaN \\
	Caffeine                           &    mg &       CAFFN \\
	Theobromine                        &    mg &      THEBRN \\
	Energy                             &    kJ &    ENERC\_KJ \\
	Sugars, total                      &     g &       SUGAR \\
	Galactose                          &     g &        GALS \\
	Fiber, total dietary               &     g &       FIBTG \\
	Calcium, Ca                        &    mg &          CA \\
	Iron, Fe                           &    mg &          FE \\
	Magnesium, Mg                      &    mg &          MG \\
	Phosphorus, P                      &    mg &           P \\
	Potassium, K                       &    mg &           K \\
	Sodium, Na                         &    mg &         NaN \\
	Zinc, Zn                           &    mg &          ZN \\
	Copper, Cu                         &    mg &          CU \\
	Fluoride, F                        &    µg &         FLD \\
	Manganese, Mn                      &    mg &          MN \\
	Selenium, Se                       &    µg &          SE \\
	Vitamin A, IU                      &    IU &     VITA\_IU \\
	Retinol                            &    µg &       RETOL \\
	Vitamin A, RAE                     &    µg &    VITA\_RAE \\
	Carotene, beta                     &    µg &       CARTB \\
	Carotene, alpha                    &    µg &       CARTA \\
	Vitamin E (alpha-tocopherol)       &    mg &      TOCPHA \\
	Vitamin D                          &    IU &        VITD \\
	Vitamin D2 (ergocalciferol)        &    µg &      ERGCAL \\
	Vitamin D3 (cholecalciferol)       &    µg &      CHOCAL \\
	Vitamin D (D2 + D3)                &    µg &        VITD \\
	Cryptoxanthin, beta                &    µg &       CRYPX \\
	Lycopene                           &    µg &       LYCPN \\
	Lutein + zeaxanthin                &    µg &     LUT+ZEA \\
	Tocopherol, beta                   &    mg &      TOCPHB \\
	Tocopherol, gamma                  &    mg &      TOCPHG \\
	Tocopherol, delta                  &    mg &      TOCPHD \\
	Tocotrienol, alpha                 &    mg &      TOCTRA \\
	Tocotrienol, beta                  &    mg &      TOCTRB \\
	Tocotrienol, gamma                 &    mg &      TOCTRG \\
	Tocotrienol, delta                 &    mg &      TOCTRD \\
	Vitamin C, total ascorbic acid     &    mg &        VITC \\
	Thiamin                            &    mg &        THIA \\
	Riboflavin                         &    mg &        RIBF \\
	Niacin                             &    mg &         NIA \\
	Pantothenic acid                   &    mg &      PANTAC \\
	Vitamin B-6                        &    mg &      VITB6A \\
	Folate, total                      &    µg &         FOL \\
	Vitamin B-12                       &    µg &      VITB12 \\
	Choline, total                     &    mg &       CHOLN \\
	Menaquinone-4                      &    µg &         MK4 \\
	Dihydrophylloquinone               &    µg &      VITK1D \\
	Vitamin K (phylloquinone)          &    µg &       VITK1 \\
	Folic acid                         &    µg &       FOLAC \\
	Folate, food                       &    µg &       FOLFD \\
	Folate, DFE                        &    µg &      FOLDFE \\
	Betaine                            &    mg &        BETN \\
	Tryptophan                         &     g &       TRP\_G \\
	Threonine                          &     g &       THR\_G \\
	Isoleucine                         &     g &       ILE\_G \\
	Leucine                            &     g &       LEU\_G \\
	Lysine                             &     g &       LYS\_G \\
	Methionine                         &     g &       MET\_G \\
	Cystine                            &     g &       CYS\_G \\
	Phenylalanine                      &     g &       PHE\_G \\
	Tyrosine                           &     g &       TYR\_G \\
	Valine                             &     g &       VAL\_G \\
	Arginine                           &     g &       ARG\_G \\
	Histidine                          &     g &     HISTN\_G \\
	Alanine                            &     g &       ALA\_G \\
	Aspartic acid                      &     g &       ASP\_G \\
	Glutamic acid                      &     g &       GLU\_G \\
	Glycine                            &     g &       GLY\_G \\
	Proline                            &     g &       PRO\_G \\
	Serine                             &     g &       SER\_G \\
	Hydroxyproline                     &     g &         HYP \\
	Vitamin E, added                   &    mg &         NaN \\
	Vitamin B-12, added                &    µg &         NaN \\
	Cholesterol                        &    mg &       CHOLE \\
	Fatty acids, total trans           &     g &       FATRN \\
	Fatty acids, total saturated       &     g &       FASAT \\
	4:0                                &     g &        F4D0 \\
	6:0                                &     g &        F6D0 \\
	8:0                                &     g &        F8D0 \\
	10:0                               &     g &       F10D0 \\
	12:0                               &     g &       F12D0 \\
	14:0                               &     g &       F14D0 \\
	16:0                               &     g &       F16D0 \\
	18:0                               &     g &       F18D0 \\
	20:0                               &     g &       F20D0 \\
	18:1 undifferentiated              &     g &       F18D1 \\
	18:2 undifferentiated              &     g &       F18D2 \\
	18:3 undifferentiated              &     g &       F18D3 \\
	20:4 undifferentiated              &     g &       F20D4 \\
	22:6 n-3 (DHA)                     &     g &       F22D6 \\
	22:0                               &     g &       F22D0 \\
	14:1                               &     g &       F14D1 \\
	16:1 undifferentiated              &     g &       F16D1 \\
	18:4                               &     g &       F18D4 \\
	20:1                               &     g &       F20D1 \\
	20:5 n-3 (EPA)                     &     g &       F20D5 \\
	22:1 undifferentiated              &     g &       F22D1 \\
	22:5 n-3 (DPA)                     &     g &       F22D5 \\
	Phytosterols                       &    mg &      PHYSTR \\
	Stigmasterol                       &    mg &       STID7 \\
	Campesterol                        &    mg &       CAMD5 \\
	Beta-sitosterol                    &    mg &      SITSTR \\
	Fatty acids, total monounsaturated &     g &        FAMS \\
	Fatty acids, total polyunsaturated &     g &        FAPU \\
	15:0                               &     g &       F15D0 \\
	17:0                               &     g &       F17D0 \\
	24:0                               &     g &       F24D0 \\
	16:1 t                             &     g &      F16D1T \\
	18:1 t                             &     g &      F18D1T \\
	22:1 t                             &     g &      F22D1T \\
	18:2 t not further defined         &     g &         NaN \\
	18:2 i                             &     g &         NaN \\
	18:2 t,t                           &     g &     F18D2TT \\
	18:2 CLAs                          &     g &    F18D2CLA \\
	24:1 c                             &     g &      F24D1C \\
	20:2 n-6 c,c                       &     g &    F20D2CN6 \\
	16:1 c                             &     g &      F16D1C \\
	18:1 c                             &     g &      F18D1C \\
	18:2 n-6 c,c                       &     g &    F18D2CN6 \\
	22:1 c                             &     g &      F22D1C \\
	18:3 n-6 c,c,c                     &     g &    F18D3CN6 \\
	17:1                               &     g &       F17D1 \\
	20:3 undifferentiated              &     g &       F20D3 \\
	Fatty acids, total trans-monoenoic &     g &      FATRNM \\
	Fatty acids, total trans-polyenoic &     g &      FATRNP \\
	13:0                               &     g &       F13D0 \\
	15:1                               &     g &       F15D1 \\
	18:3 n-3 c,c,c (ALA)               &     g &    F18D3CN3 \\
	20:3 n-3                           &     g &     F20D3N3 \\
	20:3 n-6                           &     g &     F20D3N6 \\
	20:4 n-6                           &     g &     F20D4N6 \\
	18:3i                              &     g &         NaN \\
	21:5                               &     g &       F21D5 \\
	22:4                               &     g &       F22D4 \\
	18:1-11 t (18:1t n-7)              &     g &    F18D1TN7 \\
	\caption{Table of all reportable nutrients as identified in the USDA Nutrition Database.\label{App:NutTable}}
\end{longtable}

\end{document}
